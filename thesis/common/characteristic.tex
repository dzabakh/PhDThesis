{\aim} В данной работе рассмотрены некоторые эксперименты с применением метода М-последовательности измерения импульсного отклика трассы распространения акустического сигнала. Среди рассмотренных задач измерение угловой зависимости коэффициента отражения акустического сигнала от поглощающего материала (глава 2), изучение прохождения акустического сигнала через струю (глава 3), а также экранирование акустического сигнала поверхностью узкого конуса (глава 4).

Рассматриваемые задачи объединены методом М-последовательности (Maximum Length Sequence или MLS), применяемым при их экспериментальном решении. Этот метод не является новым и ранее широко использовался в решении задач архитектурной акустики, где псевдошумовой сигнал может заменять звуковой импульс при измерении времени реверберации, и в радиолокации, где широкополосный лоцирующий сигнал позволяет уменьшить чувствительность к внешнему паразитному шуму. В данной работе метод М-последовательности адаптируется и применяется для разноплановых задач теории дифракции, и подтверждается его эффективность в задачах, требующих от измеренных сигналов хорошего соотношения сигнал/шум.

Кратко сформулируем основные цели работы:

1. Разработать метод измерения угловой зависимости коэффициента поглощения звукопоглощающего материала, работающий при помощи обращения интеграла Фурье-Бесселя.

2. С помощью метода М-последовательности исследовать влияние воздушной струи на проходящий через нее акустический сигнал,.

3. Обобщить результаты работы \cite{Shanin2011} на случай конуса с импедансной поверхностью и провести эксперимент, подтверждающий параболическую теорию.

{\novelty} Метод М-последовательности ранее применялся к дифракционному акустическому эксперименту (например, \cite{ValeraPhdthesis} в задаче дифракции на трехгранном конусе). 

{\reliability} Достоверность полученных результатов обеспечивается сравнением результата эксперимента с теоретическими результатами (теорией пористых сред Био в случае отражения от плоского слоя, теорией Блохинцева акустических волн в движущихся средах в случае прохождения сигнала через поток, теорией параболического уравнения и теорией Смышляева в случае дифракции на конусе).

{\probation} 
Основные результаты диссертации докладывались на следующих конференциях:

1. Четвертая открытая всероссийская конференция по аэроакустике, Звенигород, Россия, 29 сентября - 1 ноября 2015.

2. Акустика среды обитания, Москва, Россия, 13 мая 2016.

3. XV Всероссийская школа-семинар «Волновые явления в неоднородных средах» имени А.П. Сухорукова («Волны-2016»), Красновидово, Московская область, Россия, 5-10 июня 2016.

4. Четвертый международный семинар «Вычислительный эксперимент в аэроакустике», Светлогорск, Россия, 21 сентября - 24 ноября 2016.

5. Дни дифракции 2017, Санкт-Петербург, Россия, 19-23 июня 2017.

6. Дни дифракции 2018, Санкт-Петербург, Россия, 4-8 июня 2018.

7. VII Всероссийская конференция «Вычислительный эксперимент в аэроакустике», Светлогорск, Россия, 17 - 22 сентября 2018.

{\contributionTXT} 
Содержание диссертации и основные положения, выносимые на защиту, отражают персональный вклад автора в опубликованные работы. Подготовка к публикации полученных результатов проводилась совместно с соавторами, причем вклад диссертанта был определяющим. Все представленные в диссертации результаты получены лично автором или при его непосредственном участии.

Автор принимал активное участие в НИР, проводимых на кафедре акустики физического факультета МГУ имени М.В. Ломоносова:

1. Исследования новых абсорбционных материалов (24 декабря 2015 - 24 декабря 2016);

2. Разработка методики высокоточных дифракционных экспериментов (1 июня 2015 - 15 октября 2015);

3. Развитие новых экспериментальных и теоретических методов исследования звуковых полей и применение этих методов к задачам архитектурной акустики (1 января 2014 - 31 декабря 2016);

\textbf{Публикации.} Материалы диссертации опубликованы в 8 печатных работах, из них: 2 статьи в рецензируемых журналах (из них 1 только принята к публикации), 3 статьи в трудах конференций (из них 1 только принята к публикации), 4 в тезисах докладов.

\textbf{Структура и объем диссертации.} Диссертация состоит из введения, обзора литературы, четырех глав, заключения, приложения и библиографии. Общий объем диссертации 80 страниц, включающих 30 рисунков. Библиография включает 70 наименований на 5 страницах.

%{\actuality} 
%Обзор, введение в тему, обозначение места данной работы в
%мировых исследованиях и~т.\:п., можно использовать ссылки на~другие
%работы\ifnumequal{\value{bibliosel}}{1}{~\autocite{Gosele1999161}}{}
%(если их~нет, то~в~автореферате
%автоматически пропадёт раздел <<Список литературы>>). Внимание! Ссылки
%на~другие работы в разделе общей характеристики работы можно
%использовать только при использовании \verb!biblatex! (из-за технических
%ограничений \verb!bibtex8!. Это связано с тем, что одна
%и~та~же~характеристика используются и~в~тексте диссертации, и в
%автореферате. В~последнем, согласно ГОСТ, должен присутствовать список
%работ автора по~теме диссертации, а~\verb!bibtex8! не~умеет выводить в одном
%файле два списка литературы).
%При использовании \verb!biblatex! возможно использование исключительно
%в~автореферате подстрочных ссылок
%для других работ командой \verb!\autocite!, а~также цитирование
%собственных работ командой \verb!\cite!. Для этого в~файле
%\verb!Synopsis/setup.tex! необходимо присвоить положительное значение
%счётчику \verb!\setcounter{usefootcite}{1}!.

%Для генерации содержимого титульного листа автореферата, диссертации
%и~презентации используются данные из файла \verb!common/data.tex!. Если,
%например, вы меняете название диссертации, то оно автоматически
%появится в~итоговых файлах после очередного запуска \LaTeX. Согласно
%ГОСТ 7.0.11-2011 <<5.1.1 Титульный лист является первой страницей
%диссертации, служит источником информации, необходимой для обработки и
%поиска документа>>. Наличие логотипа организации на~титульном листе
%упрощает обработку и~поиск, для этого разметите логотип вашей
%организации в папке images в~формате PDF (лучше найти его в векторном
%варианте, чтобы он хорошо смотрелся при печати) под именем
%\verb!logo.pdf!. Настроить размер изображения с логотипом можно
%в~соответствующих местах файлов \verb!title.tex!  отдельно для
%диссертации и автореферата. Если вам логотип не~нужен, то просто
%удалите файл с~логотипом.

%\ifsynopsis
%Этот абзац появляется только в~автореферате.
%Для формирования блоков, которые будут обрабатываться только в~автореферате,
%заведена проверка условия \verb!\!\verb!ifsynopsis!.
%Значение условия задаётся в~основном файле документа (\verb!synopsis.tex! для
%автореферата).
%\else
%Этот абзац появляется только в~диссертации.
%Через проверку условия \verb!\!\verb!ifsynopsis!, задаваемого в~основном файле
%документа (\verb!dissertation.tex! для диссертации), можно сделать новую
%команду, обеспечивающую появление цитаты в~диссертации, но~не~в~автореферате.
%\fi

% {\progress} 
% Этот раздел должен быть отдельным структурным элементом по
% ГОСТ, но он, как правило, включается в описание актуальности
% темы. Нужен он отдельным структурынм элемементом или нет ---
% смотрите другие диссертации вашего совета, скорее всего не нужен.

%{\aim} 
%данной работы является \ldots

%Для~достижения поставленной цели необходимо было решить следующие 
%{\tasks}:
%\begin{enumerate}
%  \item Исследовать, разработать, вычислить и~т.\:д. и~т.\:п.
%  \item Исследовать, разработать, вычислить и~т.\:д. и~т.\:п.
%  \item Исследовать, разработать, вычислить и~т.\:д. и~т.\:п.
%  \item Исследовать, разработать, вычислить и~т.\:д. и~т.\:п.
%\end{enumerate}


%{\novelty}
%\begin{enumerate}
%  \item Впервые \ldots
%  \item Впервые \ldots
%  \item Было выполнено оригинальное исследование \ldots
%\end{enumerate}

%{\influence} \ldots

%{\methods} \ldots

%{\defpositions}
%\begin{enumerate}
%  \item Первое положение
%  \item Второе положение
%  \item Третье положение
%  \item Четвертое положение
%\end{enumerate}
%В папке Documents можно ознакомиться в решением совета из Томского ГУ
%в~файле \verb+Def_positions.pdf+, где обоснованно даются рекомендации
%по~формулировкам защищаемых положений. 

%{\reliability} 
%полученных результатов обеспечивается \ldots \ Результаты находятся в соответствии с результатами, полученными %другими авторами.


%{\probation}
%Основные результаты работы докладывались~на:
%перечисление основных конференций, симпозиумов и~т.\:п.

%{\contribution} Автор принимал активное участие \ldots

%\publications\ Основные результаты по теме диссертации изложены в ХХ печатных изданиях~\cite{Sokolov,Gaidaenko,Lermontov,Management},
%Х из которых изданы в журналах, рекомендованных ВАК~\cite{Sokolov,Gaidaenko}, 
%ХХ --- в тезисах докладов~\cite{Lermontov,Management}.

%\ifnumequal{\value{bibliosel}}{0}{% Встроенная реализация с загрузкой файла через движок bibtex8
%    \publications\ Основные результаты по теме диссертации изложены в XX печатных изданиях, 
%    X из которых изданы в журналах, рекомендованных ВАК, 
%    X "--- в тезисах докладов.%
%}{% Реализация пакетом biblatex через движок biber
%Сделана отдельная секция, чтобы не отображались в списке цитированных материалов
%    \begin{refsection}[vak,papers,conf]% Подсчет и нумерация авторских работ. Засчитываются только те, которые были прописаны внутри \nocite{}.
        %Чтобы сменить порядок разделов в сгрупированном списке литературы необходимо перетасовать следующие три строчки, а также команды в разделе \newcommand*{\insertbiblioauthorgrouped} в файле biblio/biblatex.tex
%        \printbibliography[heading=countauthorvak, env=countauthorvak, keyword=biblioauthorvak, section=1]%
%        \printbibliography[heading=countauthorconf, env=countauthorconf, keyword=biblioauthorconf, section=1]%
%        \printbibliography[heading=countauthornotvak, env=countauthornotvak, keyword=biblioauthornotvak, section=1]%
%        \printbibliography[heading=countauthor, env=countauthor, keyword=biblioauthor, section=1]%
%        \nocite{%Порядок перечисления в этом блоке определяет порядок вывода в списке публикаций автора
%                vakbib1,vakbib2,%
%                confbib1,confbib2,%
%                bib1,bib2,%
%        }%
%        \publications\ Основные результаты по теме диссертации изложены в~\arabic{citeauthor}~печатных изданиях, 
%        \arabic{citeauthorvak} из которых изданы в журналах, рекомендованных ВАК, 
%        \arabic{citeauthorconf} "--- в~тезисах докладов.
%    \end{refsection}
%    \begin{refsection}[vak,papers,conf]%Блок, позволяющий отобрать из всех работ автора наиболее значимые, и только их вывести в автореферате, но считать в блоке выше общее число работ
%        \printbibliography[heading=countauthorvak, env=countauthorvak, keyword=biblioauthorvak, section=2]%
%        \printbibliography[heading=countauthornotvak, env=countauthornotvak, keyword=biblioauthornotvak, section=2]%
%        \printbibliography[heading=countauthorconf, env=countauthorconf, keyword=biblioauthorconf, section=2]%
%        \printbibliography[heading=countauthor, env=countauthor, keyword=biblioauthor, section=2]%
%        \nocite{vakbib2}%vak
%        \nocite{bib1}%notvak
%        \nocite{confbib1}%conf
%    \end{refsection}
%}
%При использовании пакета \verb!biblatex! для автоматического подсчёта
%количества публикаций автора по теме диссертации, необходимо
%их~здесь перечислить с использованием команды \verb!\nocite!.
