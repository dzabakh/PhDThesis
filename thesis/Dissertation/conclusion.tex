\chapter*{Заключение}                       % Заголовок
\addcontentsline{toc}{chapter}{Заключение}  % Добавляем его в оглавление

%% Согласно ГОСТ Р 7.0.11-2011:
%% 5.3.3 В заключении диссертации излагают итоги выполненного исследования, рекомендации, перспективы дальнейшей разработки темы.
%% 9.2.3 В заключении автореферата диссертации излагают итоги данного исследования, рекомендации и перспективы дальнейшей разработки темы.
%% Поэтому имеет смысл сделать эту часть общей и загрузить из одного файла в автореферат и в диссертацию:

Основные результаты работы заключаются в следующем.

%% Согласно ГОСТ Р 7.0.11-2011:
%% 5.3.3 В заключении диссертации излагают итоги выполненного исследования, рекомендации, перспективы дальнейшей разработки темы.
%% 9.2.3 В заключении автореферата диссертации излагают итоги данного исследования, рекомендации и перспективы дальнейшей разработки темы.
Описан метод проведения прямого дифракционного эксперимента на основе М-последовательности. Частью методики является восстановление объемной скорости монопольного источника акустических волн с помощью метода двух микрофонов. Скорость восстаналивается с помощью теории Вайнштейна об излучении из открытого конца волновода. С помощью описанной методики:
\begin{enumerate}
	\item был проведен эксперимент по измерению угловой зависимости коэффициента отражения от звукопоглощающего материала, который позволит получить исчерпывающую информацию о звукоотражающих свойствах строительных материалов;
	\item было проведено исследование прохождения звука через воздушную струю со сравнением результатов с численной моделью;
	\item был проведен эксперимент по рассеянию акустических волн на узком конусе со сравнением результатов эксперимента с численной моделью на основе параболического уравнения теории дифракции.
\end{enumerate}

В заключение автор выражает благодарность и большую признательность научному руководителю Шанину А.В. за поддержку, помощь, обсуждение результатов и научное руководство. Также автор благодарит Королькова А.И. за помощь в теоретических расчетах и плодотворные дискуссии.